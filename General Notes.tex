\documentclass{article}
\usepackage{geometry}
\usepackage[utf8]{inputenc}
\usepackage{graphicx}
\graphicspath{ {images/} }
\geometry{a4paper, left=25mm, right=25mm, top=25mm, bottom=25mm}
\setlength{\parindent}{2em}
\renewcommand{\baselinestretch}{1.3}
% to add header and footer information
\usepackage{fancyhdr}
\pagestyle{fancy}
\fancyhf{}
\rhead{S.Goldie - 42611814}
\lhead{FOAR705 - Learning Journal}
\lfoot{Session 2, 2019 - Macquarie University}
\rfoot{Page \thepage}
% to create lists with nested bullet points
\usepackage{outlines}
% To hyperlink references in contents and other lists
\usepackage[colorlinks=true,linkcolor=blue]{hyperref}
\usepackage{hyperref}
% To create a list of labels
% https://tex.stackexchange.com/a/418302/5482
\usepackage{crossreftools}

\title{FOAR705 - General Notes}
\author{Sheriden Goldie}
\date{}

\begin{document}

\maketitle

% create table of contents based on sections/subsections
\tableofcontents

% Create and index of errors
% so the list has a pretty name
\renewcommand
\listoflabelsname{List of Errors}

\crtlistoflabels


\section*{General Notes by Week}
\section{29th July - 4th August}
Before beginning this class I was very apprehensive about the content and what relevance it would have to my thesis and research. As a creative writing student many of my processes are manual by necessity and the creative process is one that is difficult to automate. 

After the first class, I felt both more at ease and more unsure. The teaching staff were friendly and approachable, although also nervous about what this course will offer creative practice students like myself. I am reassured that at least we will work through the issues together, and I am not alone in my concerns or challenges. 

The class will be using Slack for communication, GitHub for data repositories, Cloudstor for data/file sharing, and Overleaf/LaTeX for document creation (for those willing).

\subsection{Responses to Week 1 Exercises}
\textbf{Data Retrieval}

We were tasked to retrieve a file from at least 6 months prior. 
To do this I opened my Document folder using Windows Explorer and navigated to a file from my undergraduate studies.
However I realised that this is not in fact a back up. This information wasn't in any way archived or protected - meaning that if anything were to happen to my desktop computer, this data could easily be lost.
I use OneDrive - and while initially I thought this was a backup, I now realise it is in actual fact a cloud server, that I use for convenience. It allows me to access files from both my desktop computer, my laptop and my phone. However, again this is at the mercy of Microsoft services. Should that provider change, or have issues, that information is also vulnerable. 

I read the article shared by Brian on Slack about data backup plans, and resolved to implement a weekly system for my university and creative projects at the very least. 
\\*
\\*
\noindent \textbf{Tedious Process in Current Work}

Currently I am unsure of what tedious processes to include here.

In my work getting started is a tiring task. Organising notes, ideas and inspiration into something I can work with for a creative project is difficult and time consuming.

References and citations are only mildly troublesome. I use Scrivener for most of my projects - and set up bibliographies as I progress through the work. For a 3000 word essay it isn't too difficult to keep track of various trains of thoughts in my head. I can see that this will pose a problem for a larger project - like the thesis.

\section{5 - 11th August}

This week we began work on Data Carpentry. We only got through the introduction section in class - so this is left to finish in our own time. I will work with Mona on this on Wednesday in Week 3. Hopefully I can help her, as she is not confident using computers. 

We will also have the scoping exercise due next week. The information for this will be posted/uploaded to Cloudstor, as well as iLearn.

\section{12 - 18th August}

15th August - completed a part of the Data Carpentry Module - Spreadsheets for Social Sciences

This week we also have the scoping exercise/proof of concept assignment due. I submitted this on Monday - as I realised I could run short on time later in the week. 
I used LaTeX to create the document, and submitted it as a PDF to iLearn, as well as a .TEX document on Cloudstor. 
I didn't have any issue creating the document, and I added header and footer features, as well as a table of contents. These features are a little redundant - but it was an opportunity  to test the media. 

I met with Mona this week to help her better understand the class and all the different things that need to be done. I have also completed the Data Carpentry exercises, and created a separate learning journal entry for that experience. 

\subsection{LaTeX and Slack}
\label{ LaTeX Alignment}
I also shared some frustrations/findings with the LaTeX channel on Slack that I started - this was to do with alignments of text in a document. O was able to solve the issue on my own by looking up indents on the LaTeX learn site - but hopefully it will be useful for others to see the process/outcome.

I have also worked to consolidate my favourite LaTeX packages into a template document for easy access with notes - ``metadata'' if you will - to enable me to reuse them - and remember what they do. 

\subsection{Questions to raise with Brian/Shawn during the consultation time before/after class:}
\label{ Issues raised with Brain/Shawn Week 3}
\begin{outline}[enumerate]
    \1 Combining GitHub Repositories (is it possible to set up a GitHub repository then sync to it from overleaf)
        \2 Potentially archive unused repository
        \2 check with Brian - he can give ownership of repo to me to enable deletion if needed
    \1 Verbatim text in Overleaf/LaTeX - how to have this occur in-line
        \2 This is determined by the document type, and it seems to actually be a better practice in terms of clear documentation to have this continue to be formatted as the document dictates
    \1 Line breaks in Overleaf/LaTeX - I think I might be using different terminology - hence not being able to find a satisfactory answer
        \2 Look up:
            \3 Line break, or carriage return, or paragraph
            \3 I worked with Kathryn on this and we were able to find a solution using the code:
    \begin{verbatim}
        \\*
        \\*
    \end{verbatim}
    \1 Check in on assignment feasibility - indexing note taking system/maybe something that can generate something to link to LaTeX for referencing purposes?
        \2 Look up - this research also adds to Elaboration Assignments
            \3 Tropy - works with scanned archival documents. 
            \3 Content management systems - Omeka
            \3 CMS - content management systems
            \3 API intermediary step - using opensource software then creating a modification
            \3 Possible alternative to note taking indexing plan - Utilising Brian's work on static websites to develop a new model for `The Quarry' website.
    \1 Do I need to separate my Data Carpentry exercises into a separate repository to make them easier to see?
        \2 Feedback from Shawn - No, as part of the Learning Journal is fine. They will be submitted as part of the learning journal. Submit to iLearn as zip file if necessary.
\end{outline}
    
\subsection{Data and Metadata in English Literature Studies}

Data has been a difficult term to lock down in literary studies, and similar fields like philosophy. We do not have the same quantifiable data like science, or social sciences. 

It could be useful to consider the texts that we read to then write about as data. We can then treat/manipulate that data in a certain way (close reading, distant reading), which informs how we interpret that data and draw new conclusions from that process. 

This also means that our data has metadata, such as page numbers - the ability to find the section of text that your might reference. The author, title, and date of publication, are also examples of metadata.

\section{19 - 25th August}

This week the Scoping Part 2 is due, which is a decomposition of the ideas brought forward in the first scoping exercise. 
We are also required to finish the Spreadsheet for Social Sciences learning material on DataCarpentry.org

We will also submit our learning journal as completed thus far for feedback. 

I began investigating citation/reference management software, and CMS Software. 

I need to add an Error's List to my learning journal

\subsection{Mendeley:}
I set out to utilise Mendeley to aid with reference management for a short essay I was writing for an English Literature Class. 
I was using this task to test the Mendeley system, and gauge it's feasibility for further use on other projects. 

I am not enthused with this system so far, but it has some useful features, like being able to sync resources across devices. 

I do not like the localised software interface, the text is very small on my screen, and i have yet to figure out how to adjust that. The online interface is more legible. 

It is handy to have a space for notes on each document/source - I have used this to record Macquarie University call numbers for some books I had borrowed. 

When using the Microsoft Word plugin, I enjoyed the ease of which the bibliography was generated, but I disliked how fiddly the in-text citations were. They would put in the authors name, but I had to manually enter the page number for the reference, with repeated checks that the manual change was what I wanted to proceed with. This also allows for errors to be made if one is not careful. 

Ultimately I was able to complete the task I set out to do, however I feel that I have not explored the full capabilities of this software yet. 
I will attempt using this software more comprehensively at a later date.

\subsection{Questions to raise with Brian/Shawn}
\begin{outline}[enumerate]
    \1 Mendeley - Native screen resolution issue
        \2 Check DPI scaling - and then settings on application on whether it is conforming to system DPI scaling settings
    \1 Addressing the issue of a combined error list - across multiple documents? relevance/feasibility 
        \2 Brian is looking into a proof of concept
    \1 Book recommendation: Rainbows End by Vernor Vinge
\end{outline}

I also found: https://nickblackbourn.com/

He uses propriety software to achieve a research and writing workflow that closely resembles what I am aspiring to create and implement for myself, so it is a good starting point. I hope that by using opensource software I can implement a similar, or even improved, system for my own research.

\section{26th August - 1st September}

\label{LaTeX Label Error}
\textbf{Latex Label Error}

I found that when applying the crossreftool style package to LaTeX, that i had and unexpected error with an image that was inserted. This was because the Figure had a parameter also called:
\begin{verbatim}
    \label{}
\end{verbatim}

Because of this double up, the compile process was trying to use the figure label for the list of labels - but the format was unexpected. So I changed the way I added this image to a simpler inline command:
\begin{verbatim}
    \includegraphics[]{}
\end{verbatim}

Brian responded to my query about this on Slack - and a band-aid fix is sufficient for the moment.


\section{2nd - 8th September}

\section{9th - 15th September}

\section*{Semester Recess}

\section{23rd - 29th September}

\section{30th September - 6th October}

\section{7th - 13th October}

\section{14th - 20th October}

\section{21st - 27th October}

\section{28th October - 3rd November}

\section*{Final Reflections}


\end{document}