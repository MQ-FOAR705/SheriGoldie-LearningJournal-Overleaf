\documentclass{article}
\usepackage{geometry}
\usepackage[utf8]{inputenc}
\usepackage{graphicx}
\graphicspath{ {images/} }
\geometry{a4paper, left=25mm, right=25mm, top=25mm, bottom=25mm}
\setlength{\parindent}{2em}
\setlength{\parskip}{1em}
\renewcommand{\baselinestretch}{1.3}

\title{FOAR705 - S.Goldie Learning Journal}
\author{Sheriden Goldie}
\date{}

\begin{document}

\maketitle


\section*{General Notes by Week}
\section{29th July - 4th August}
Before beginning this class I was very apprehensive about the content and what relevance it would have to my thesis and research. As a creative writing student many of my processes are manual by necessity and the creative process is one that is difficult to automate. 

After the first class, I felt both more at ease and more unsure. The teaching staff were friendly and approachable, although also nervous about what this course will offer creative practice students like myself. I am reassured that at least we will work through the issues together, and I am not alone in my concerns or challenges. 

The class will be using Slack for communication, GitHub for data repositories, Cloudstor for data/file sharing, and Overleaf/LaTeX for document creation (for those willing).

\subsection{Responses to Week 1 Exercises}
\textbf{Data Retrieval}

We were tasked to retrieve a file from at least 6 months prior. 
To do this I opened my Document folder using Windows Explorer and navigated to a file from my undergraduate studies.
However I realised that this is not in fact a back up. This information wasn't in any way archived or protected - meaning that if anything were to happen to my desktop computer, this data could easily be lost.
I use OneDrive - and while initially I thought this was a backup, I now realise it is in actual fact a cloud server, that I use for convenience. It allows me to access files from both my desktop computer, my laptop and my phone. However, again this is at the mercy of Microsoft services. Should that provider change, or have issues, that information is also vulnerable. 

I read the article shared by Brian on Slack about data backup plans, and resolved to implement a weekly system for my university and creative projects at the very least. 

\textbf{Tedious Process in Current Work}

Currently I am unsure of what tedious processes to include here.

In my work getting started is a tiring task. Organising notes, ideas and inspiration into something I can work with for a creative project is difficult and time consuming.

References and citations are only mildly troublesome. I use Scrivener for most of my projects - and set up bibliographies as I progress through the work. For a 3000 word essay it isn't too difficult to keep track of various trains of thoughts in my head. I can see that this will pose a problem for a larger project - like the thesis.

\section{5 - 11th August}

This week we began work on Data Carpentry. We only got through the introduction section in class - so this is left to finish in our own time. I will work with Mona on this on Wednesday in Week 3. Hopefully I can help her, as she is not confident using computers. 

We will also have the scoping exercise due next week. The information for this will be posted/uploaded to Cloudstor, as well as iLearn.



\section{12 - 18th August}

This week we also have the scoping exercise/proof of concept assignment due. I submitted this on Monday - as I realised I could run short on time later in the week. 
I used LaTeX to create the document, and submitted it as a PDF to iLearn, as well as a .TEX document on Cloudstor. 
I didn't have any issue creating the document, and I added header and footer features, as well as a table of contents. These features are a little redundant - but it was an opportunity  to test the media. 

I also shared some frustrations/findings with the LaTeX channel on Slack that I started - this was to do with alignments of text in a document. O was able to solve the issue on my own by looking up indents on the LaTeX learn site - but hopefully it will be useful for others to see the process/outcome.

I met with Mona this week to help her better understand the class and all the different things that need to be done. I have also completed the Data Carpentry exercises, and created a separate learning journal entry for that experience. 

Questions to raise with Brian during the consultation time before class:
\begin{enumerate}
    \item Combining GitHub Repositories (is it possible to set up a GitHub repository then sync to it from overleaf)
    \item Verbatim text in Overleaf/LaTeX - how to have this occur in-line
    \item Line breaks in Overleaf/LaTeX - I think I might be using different terminology - hence not being able to find a satisfactory answer
    \item Check in on assignment feasibility - indexing note taking system/maybe something that can generate something to link to LaTeX for referencing purposes?
    \item Do i need to separate my Data Carpentry exercises into a separate repository to make them easier to see?
\end{enumerate}



\section{19 - 25th August}

\section{26th August - 1st September}

\section{2nd - 8th September}

\section{9th - 15th September}

\section*{Semester Recess}

\section{23rd - 29th September}

\section{30th September - 6th October}

\section{7th - 13th October}

\section{14th - 20th October}

\section{21st - 27th October}

\section{28th October - 3rd November}

\section*{Final Reflections}


\end{document}