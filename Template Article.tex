\documentclass{article}
\usepackage{geometry}
\usepackage[utf8]{inputenc}
\usepackage{graphicx}
% package and file page setting for image input
\graphicspath{ {images/} }

\usepackage{geometry}
% package for setting page format and size, and setting paragraph indents, paragraph spacing, and line spacing. 
\geometry{a4paper, left=25mm, right=25mm, top=25mm, bottom=25mm}
\setlength{\parindent}{2em}
\setlength{\parskip}{1em}
\renewcommand{\baselinestretch}{1.3}
% for Baseline stretch
% 1 = 1 line spacing in Word 
% 1.3 = 1.5 line spacing in Word 
% 1.5 = double spacing in Word

\usepackage{fancyhdr}
%package to include header and footer information. Only starts from second page with these settings.
\pagestyle{fancy}
\fancyhf{}
\rhead{Right Side of Header}
\lhead{Left Side of Header}
\lfoot{Left side of Footer}
\rfoot{Page \thepage}

\usepackage{dirtytalk}
% this package is useful for typesetting quotes - it allows for nested quotes using \say {xxxx \say{nested items here} XXXX. }

\usepackage{outlines}
%this package is a way to enable the display of itemised lists with nested sub-items as well. 

% To hyperlink references in contents and other lists
\usepackage[colorlinks=true,linkcolor=blue]{hyperref}
\usepackage{hyperref}
% To create a list of labels
% https://tex.stackexchange.com/a/418302/5482
\usepackage{crossreftools}
%To apply filtering function to Label List Creation - only include labels beginning with "Error:" - means figure labels won't be affected.
\newcommand{\includelabelintoc}{Error:}


\title{Title Goes Here}
\author{Author Goes Here}
\date{Date Goes Here}

\begin{document}

\maketitle

\tableofcontents
%creates a table of contents based on sections and subsections

\pagebreak
%ends text on current page, and moves all following text to the next page.

\section{Section Heading}
\subsection{Subsection Heading}

This is the main body of text. \say{This is a quote about a \say{nested quote}, so that I can test the \say{dirtytalk} package.}

This is the next paragraph. 

\textit{This is italics}

\textbf{This is bold}

This is a new paragraph with \emph{emphasized text} to make it stand out. 

\textbf{Quotation Marks}

`this is a quote using single quotation marks'

"this is a quote that looks odd"

``this is a quote with double quotation marks''
%this is the best way to deal with double quotation marks in LaTeX without using any additional packages
% the ` is found on the tilde key - ~ on my keyboard it is at the top left next to 1/!
\\*
\\*
To create a line break use the following:
\begin{verbatim}
    \\*
    \\*
\end{verbatim}

% use the above for inserting a line break - it requires the same \\* to be entered on two lines, one after the other.

\textbf{Nested List Items}

An alternative to this approach, which facilitates the production of nested lists is the outlines package. To produce a bulleted list with three levels it is as simple as

\begin{outline}
 \1 Top level item
   \2 Sub item
     \3 sub sub item
\end{outline}


To make a numbered list (as opposed to a bulleted list) one can simply pass the enumerate option to this package

\begin{outline}[enumerate]
 \1 Top level item
   \2 Sub item
     \3 sub sub item
\end{outline}


%%% How I am insertung figures: default


\begin{figure}[htbp]
    \centering
    \includegraphics[width=11cm]{}
    \caption{Caption}
    \label{fig:my_label}
\end{figure}

\end{document}
