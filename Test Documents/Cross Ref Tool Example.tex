\documentclass{article}
\usepackage[utf8]{inputenc}
\usepackage{lipsum}
\usepackage{cleveref}
%https://www.overleaf.com/learn/latex/Paragraph_formatting#Paragraph_spacing
\usepackage[utf8]{inputenc}
\usepackage[english]{babel}



\usepackage[colorlinks=true,linkcolor=blue]{hyperref}

\usepackage{hyperref}

% https://tex.stackexchange.com/a/418302/5482
\usepackage{crossreftools}

\newcommand{\includelabelintoc}{error:}
% https://tex.stackexchange.com/a/436076/5483
\usepackage{tocloft}
\advance\cftsecnumwidth 0.5em\relax
\advance\cftsubsecindent 0.5em\relax
\advance\cftsubsecnumwidth 0.5em\relax


\title{Demonstration of crossreftools}
\author{Brian Ballsun-Stanton}
\date{August 2019}

\begin{document}

\maketitle

\tableofcontents


% so the list has a pretty name
\renewcommand
\listoflabelsname{List of Errors}

% Hack to do single-line skipping in list of errors
\setlength{\parskip}{-1em}

\crtlistoflabels


% We change parskip *after* create list of labels...

%This sets the indent length of a paragraph 
\setlength{\parindent}{0em}

%This sets the whitespace between paragraphs
\setlength{\parskip}{1em}

\section{Introduction}


\lipsum[1]
\lipsum[1]
\lipsum[1]
\subsection{Specific activity 1}
\lipsum[1]
\subsection{Error 1}
\label{error: This was a silly error}
\lipsum[1]
\subsection{Specific activity 1}
\lipsum[1]
\subsection{Non-displayed Error 2}
\label{This was a different silly error that isn't displayed}
\lipsum[1]
\subsection{Displayed Error 3}
\label{error: This was a different silly error that is displayed}
\lipsum[1-3]



\end{document}